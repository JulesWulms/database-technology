\section{Introduction} \label{sec:Introduction}
Optimizing a query can be done in several ways. The operation that is the computational most expensive is the join operation. So it makes sense to start by making this operation faster. Changing the order in which joins are performed is a common approach. For this approach cost approximations are computed before actually performing the joins. This approach requires  the development of a cost model, an assignment of an estimated cost to each query processing plan and searching in the (huge) search space for the cheapest cost plan. There are also approaches that do not use a cost plans. One of the approaches focusses on minimizing the number of joins instead of using a optimal join order. This approach requires a homomorphism test, which is NP-complete. An other approach focusses on the structural properties of the queries. It will try to find a project-join order that will minimize the size of the intermediate results during query evaluation. Practical it means, use only the data that you need and throw away the rest as soon as possible. 

The paper we are reviewing is such a structural approach. The authors of that paper try to push down the projections, such that the attributes that are not needed are projected out as early as possible. The projection pushing strategy has been applied to solve constraint satisfaction problems in Artificial Intelligence with good experimental results. The input to a constraint-satisfaction problem consists of a set of variables, a set of possible values for the variables, and a set of constraints between the variables; the question is to determine whether there is an assignment of values to the variables that satisfies the given constraints. IN the database area, this means that you are searching for entries that satisfy the constraints on their attributes. 

Optimizing the query manually, often focusses on reducing the search space before the join operation. In this fashion the number of entries used in the join is less, which can result in a huge performance gain. Most of the time the projections are pushed down by selecting as soon as possible or pushing the projection to a sub query. It is possible to make a sub-query that projects out all irrelevant information and tries to reduce the intermediate results. Optimizing the query in an automated and structural way looks promising and practical. 